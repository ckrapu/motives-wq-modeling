\documentclass[12pt,letterpaper]{article}

% Packages
\usepackage[utf8]{inputenc}
\usepackage{graphicx}
\usepackage{amsmath}
\usepackage{amssymb}
\usepackage{natbib}
\usepackage{hyperref}
\usepackage{booktabs}
\usepackage{geometry}

% Page geometry
\geometry{margin=1in}

% Title information
\title{Water Quality Modeling Research}
\author{Author Name}
\date{\today}

\begin{document}

\maketitle

\begin{abstract}
    We used a recently developed national dataset on riverine nitrogen yield in 66,000 stream reaches in the continental United States to produce a hierarchical geostatistical linear model capable of predicting changes in long-term average total nitrogen due to changes in land cover and explaining X \% of the variation of the dataset. Major components of this model include sparsity-inducing priors for regression coefficients, and a novel random-effects structure represented as the sum of a Gaussian process defined on the geographic coordinates of each catchment with a Gaussian Markov Random Field to account for unmodeled variation showing strong autocorrelation between adjacent stream network segments. For applications requiring greater accuracy, We also include a second-stage machine learning model which can further enhance predictive accuracy at the expense of easy interpretability (X \% accuracy). 
\end{abstract}

\section{Introduction}
\label{sec:introduction}

\section{Methods and Data}
\label{sec:methods}

\subsection{Data}

\subsection{Modeling}

\section{Results}
\

\section{Discussion}
\
\section{Conclusion}
\label{sec:conclusion}


\section*{Acknowledgments}

\bibliographystyle{plainnat}
\bibliography{references}

\end{document}
